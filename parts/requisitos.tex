\begin{center}
	%comando \addrequisito{identificação>}{<Nome>}{<data de criação>}{<local>}{<responsável>}{<especificação>}
	\addrequisito{RF-01}
	   {Gerenciar o registro de calçados.}
	   {28/02/2018}
	   {IF Goiano Campus Ceres – Lab.03 PC.27}
	   {Andressa}
	   {Os produtos devem ser cadastrados com os seguintes dados: marca, número, categoria(ex:esportivo,casual, etc...), cor, fornecedor, material, código de barras, gênero,tipo(infantil, jovem, infanto-juvenil, adulto), quantidade em estoque. Mostrar uma listagem de produtos com marca, modelo e número. Ao clicar no produto todos os atributos devem ser mostrados. Possibilitar busca por número do calçado, modelo ou tipo}
	 \addrequisito
	   	{RF-02}
	    {Gerenciar o registro de fornecedores.}
	    {28/02/2018}
	    {IF Goiano Campus Ceres – Lab.03 PC.27}
	    {Patricia Mesquita}
	    {Permitir o cadastro de fornecedores requirindo ao usuário as seguintes informações: nome, CNPJ ou CPF, telefone, e-mail. Deve mostrar os fornecedores para o usuário em lista com todos os atributos e a busca pode ser feita através de qualquer atributo.}
		
	   \addrequisito{RF-03}
	    {Gerenciar o registro de clientes.}
	    {28/02/2018}
	    {IF Goiano Campus Ceres – Lab.03 PC.27}
	   	{Luana Queiros}
	    {O cadastro de cliente deve conter nome, endereço, e-mail (opcional), CPF, RG, gênero (opcional), número do telefone. Exibir uma lista de clientes com os seguintes atributos nome, telefone e RG, ao clicar em um registro mostrar uma listagem com todos os atributos do cliente. Permitir a busca por qualquer atributo.}
	    
	\addrequisito{RF-04}
		{Gerenciar o registro de endereços.}
		{28/02/2018}
		{IF Goiano Campus Ceres – Lab.03 PC.27}
		{Matheus Henrique Passos}    
		{Deverá ser possível cadastrar cidade, estado ,CEP, logradouro,bairro,cidade e número (opcional). O endereço deverá ser vinculado a pelo menos um cliente ou fornecedor. Os endereço serão visíveis, no registro ao qual estão vinculados.}  
	\addrequisito{RF-05}
		{Gerenciar o registro de vendas.}
		{07/03/2018}
		{IF Goiano Campus Ceres – Lab.03 PC.27}
		{Matheus Henrique Passos}
		{Ao registrar uma venda se deve informar os produtos e o cliente que está realizando a compra, o sistema deve calcular o valor total da compra, e apresentar um campo para informar o desconto, data e hora da venda devem ser preenchidas automaticamente, funcionário responsável pela venda, venda paga(sim ou não). Depois disso o programa irá requisitar a quantidade de parcelas e o tipo de pagamento, o sistema deve também perguntar se há uma entrega caso haja deve-se requisitar o cadastro da mesma. Deve ser mostrado uma listagem de vendas, onde será possível buscar por cliente e data.}
		
	\addrequisito{RF-06}
		{Gerenciar os registros das entregas.}
		{28/02/2018}
		{IF Goiano Campus Ceres – Lab.03 PC.27}
		{Matheus Henrique Passos}
		{Ao registrar o sistema deve solicitar o endereço, responsável pela entrega e nome do receptor.}
    \addrequisito{RF-07}
		{Gerenciar os registros das compras.}
		{28/02/2018}
		{IF Goiano Campus Ceres – Lab.03 PC.27}
		{Matheus Henrique Passos}
		{O sistema deve solicitar ao usuário as seguintes informações: produto, fornecedor,valor por unidade,quantidade. Preenchida o software deve sugerir um valor total que o usuário irá confirmar ou mudar antes de finalizar o cadastro.}
    \addrequisito{RF-08}
        {Gerar relatório de vendas.}
		{09/03/2018}
		{IF Goiano Campus Ceres – Lab.03 PC.27}
		{Matheus Henrique Passos}
		{O sistema deve gerar um relatório de vendas realizadas por estação do ano, as vendas irão conter a quantidade de sapatos de um determinado modelo e cor, a marca mais vendida e qual modelo foi e marca foi o mais vendido para cada sexo e faixa de idade. O relatório deve ser emitido no formato PDF}
    \addrequisito{RF-09}
        {Gerar nota fiscal NFC-e}
		{09/03/2018}
		{IF Goiano Campus Ceres – Lab.03 PC.27}
		{Matheus Henrique Passos}
		{O sistema deve gerar uma nota fiscal eletrônica contendo as informações dos produtos comprados com o nome e CPF do cliente, informações dos tributos totais, e um QRCode para consulta.}
    \addrequisito{RF-10}
        {Registro de encomendas}
		{09/03/2018}
		{IF Goiano Campus Ceres – Lab.03 PC.27}
		{Denis Vitoriano}
		{Caso o produto não esteja em estoque o funcionário da loja deve ser capaz de registrar uma ou mais encomendas para o cliente, provendo as seguintes informações para o sistema o produto que o cliente deseja, o fornecedor e o cliente que requisitou o produto,uma vez que produto for pedido ao fornecedor o usuário deve ser capaz de inserir uma data prevista para a entrega do mesmo. A encomenda terá um campo onde estará a situação da mesma}
        \addrequisito{RF-11}
        {Relatório de encomendas}
		{09/03/2018}
		{IF Goiano Campus Ceres – Lab.03 PC.27}
		{Matheus Henrique Passos}
		{O relatório de encomendas deve ser gerado todos os dias 1h antes do fim do horário comercial, caso haja alguma encomenda pendente(ainda foi pedida ao fornecedor) apenas. Nesse relatório deve conter o e-mail e telefone do fornecedor, o produto e a quantidade a ser pedida. }
\end{center}
